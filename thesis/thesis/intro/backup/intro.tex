\chapter{Physics Motivation}
 Understanding the structure of the nucleus remains as one of the biggest challenges in nuclear physics because of the complicated many-body interaction between nucleons. The independent particle shell model has its great success in description of the nuclear structure as individual nucleons moving in the average mean field, and their energy and momenta are below the Fermi energy and the Fermi momentum, i.e., $\epsilon < \epsilon_{F}$ and $k<k_{F}$. However, the theory provides limited ability to study the short range properties of nucleon-nucleon (NN) interaction and fails to describe the structure of nuclear matter beyond the saturation density. 
 
 Short Range Correlations (SRCs) provide an successful explanation of such discrepancy by examining the high momentum component of the nucleon momentum distribution at $k>k_{F}$. The attractive and repulsive potential between nucleons at short distance ($\sim 1.0$ fm) excite the nucleons from their single shell and cause the significant increase of strength to the nuclear spectral function. 

High energy electrons scattering on a nuclear target provides an essential probe to unveil the tiny structure of nuclei and nucleons. In the kinematic region of quasi-elastic (QE) scattering, the electron interacts with the target by exchanging a virtual photon which is able to couple to a meson exchanged between two nucleons. The measurement of such process is able to resolve the nucleon-nucleon (NN) interaction and provide an effective way to study the structure of nuclear. Early experiments in SLAC conformed the evidence of SRCs and recent experiments in JLab extended the study to map out the strength of SRCs in a wider range of nuclei and examined the isospin effect in SRCs. 

 In this chapter, a review of QE electron-nucleus scattering will be given in Section 1.1, followed by a discussion of theoretical understandings of nuclear properties including the IPSM in Section 1.2 and SRCs in Section~1.3. The experimental techniques and results to explore SRCs will be briefly reviewed, and the new experiment, E08-014, will be introduced at the end.

\input intro/qe.tex
\input intro/nuclear_structure.tex
\input intro/src.tex
