\chapter{Introduction}
The basic structure of an atom is understood to be electrons orbiting around a dense central nucleus due to the attractive electromagnetic force. Scattering experiments discovered that the nucleus is further composed of nucleons which include protons with positive charges and electrically neutral neutrons. Due to the Pauli Principle and the long range property of the nucleon-nucleon (NN) interactions, nucleons act like independent particles moving in a mean field inside the nucleus, and their features, e.g. their ground-states, were successfully predicted by the independent particle shell model (IPSM). However, it became clear in the early 1970s via electron scattering experiments that to completely explain high momentum components in the nuclear wave-functions, the short-range NN interactions must be accounted for. Short range correlations (SRC) arise from the tensor component and the repulsive hard-core in the NN interactions, and are essentially important to fully understand the nuclear structure and the properties of nucleons. 

  High energy electron scattering on nuclear targets is used as a probe to unveil the structure of nuclei and nucleons. In this chapter, a brief review of nuclear structure will be given, followed by a discussion of quasielastic (QE) electron scattering.
 
\input intro/nuclear_structure.tex
\input intro/quasielastic.tex
