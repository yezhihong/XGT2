\begin{flushleft}
•
\end{flushleft}\documentclass[a4paper,10.5pt]{report}
%\documentclass[a4paper,10.5pt]{article}
\makeindex
\usepackage{booktabs}
\usepackage{epsfig}
\usepackage{graphicx}
\usepackage{epstopdf}
\usepackage{amsmath}
\usepackage{rotating}
\usepackage{caption}
\usepackage{subfig}

% Title Page
\textwidth 15 true cm
\textheight 25 true cm
%\headheight  14pt
%\headsep    16pt
%\footskip   27pt
%\marginparsep 10pt
%\marginparwidth  100pt
\def\marginset#1#2{
\setlength{\oddsidemargin}{#1}
\iffalse
\reversemarginpar
\addtolength{\oddsidemargin}{\marginparsep}
\addtolength{\oddsidemargin}{\marginparwidth}
\fi

  \setlength{\evensidemargin}{0mm}
\iffalse
\addtolength{\evensidemargin}{\marginparsep}
\addtolength{\evensidemargin}{\marginparwidth}
\fi

  % \paperwidth = h + \oddsidemargin+\textwidth+\evensidemargin + h
\setlength{\hoffset}{\paperwidth}
\addtolength{\hoffset}{-\oddsidemargin}
\addtolength{\hoffset}{-\textwidth}
\addtolength{\hoffset}{-\evensidemargin}
\setlength{\hoffset}{0.5\hoffset}
\addtolength{\hoffset}{-1in}           % h = \hoffset + 1in

  \setlength{\voffset}{-1in}             % 0 = \voffset + 1in
\setlength{\topmargin}{\paperheight}
\addtolength{\topmargin}{-\headheight}
\addtolength{\topmargin}{-\headsep}
\addtolength{\topmargin}{-\textheight}
\addtolength{\topmargin}{-\footskip}
\addtolength{\topmargin}{#2}
\setlength{\topmargin}{0.5\topmargin}
}

\marginset{10mm}{12mm}
\title{E08014 Cross Section Extraction}
%\subtitle{A note of $x>2$ cross section extraction package}
\author{Zhihong Ye\\ University of Virginia}

\begin{document}
%\maketitle

\section{Yield Ratio Method:} 
The following results are obtained by using Yield Ratio Method to extract Cross Sections. Comparing Yield distribution and the Yield Ratio at each kinematics setting are also given.

The Yield of Experiment Data is written as:
\begin{equation}
   Y^{i}_{EX} = \frac{N^{i}_{EX}}{N_{e} \cdot \epsilon_{eff}} ,
 \label{eqyex}
\end{equation}
where $N^{i}_{EX}$ is the total number of events in the $ith$ $x_{bj}$ bin after all cuts, $N_{e}$ is the total charge for all runs and $\epsilon_{eff}$ is the total efficiency, including detection and PID cuts, and the values are set to one as we discussed earlier. A factor of $10^{33}$ is applied on this yield to match the unit of Monte Carlo Yield.

And the Yield of Monte Carlo data is given by:
\begin{equation}
   Y^{i}_{MC} = \frac{N_{tg}\cdot \sum_{j\in i}\sigma^{rad}_{model}(E'_{j},\theta_{j})\cdot \Delta\Omega_{MC} \Delta E'_{MC}}{N_{MC}^{gen}} ,
   \label{eqymc}
\end{equation}
where $N_{tg}$ is the total scattering centers of the target; $\sum_{j\in i}$ means sumarzing all Monte Carlo events in the $ith$ $x_{bj}$ bin, and each event has its radiated cross section value $\sigma(E'_{j},\theta_{j})$ calculating from the XEMC model (No DIS part in radiation tail).  $\Delta\Omega_{MC} \Delta E'_{MC}$ is the full phase space in the Monte Carlo and $N_{MC}^{gen}$ is the total generated events (Normally 2M).

So the experimental Born cross section can be extracted by taking the ratio and timing the born cross section from the XEMC model:
\begin{equation}
   \sigma^{born}_{EX}(x_{bj}^{i}) = \frac{ Y^{i}_{EX}}{Y^{i}_{MC}} \cdot \sigma^{born}_{model}(E'_{i}, \theta_{0}),
   \label{eqxs}
\end{equation}
where $E'_{i}$ is the scattered momentum directly calculated by using the central value of the $ith$ $x_{bj}$ bin and the central scattering $\theta_{0}$, and beam energy fixed at 3.356 GeV.


\section{Statistic Errors Calculation}
The detail of Statistic Errors are calculated as following:
\begin{enumerate}
%\item \textbf{$N_{tg}$:}  Form $N_{tg} = \frac{\rho\cdot l \cdot N_{a}}{A}$, and $\rho_{cor} = \rho \cdot (1.0 - B \cdot I /100)$, there are three terms that can introduce errors: beam current measurement and calculation ($\delta I$), acuracy of Boiling Factors ($\delta B$), and the acuracy of target thickness measurement ($\delta \rho$). Last term is known but I temperately set the first two terms to zero. Hence:
%\begin{equation}
%  \delta N_{tg} = \frac{\delta\rho}{\rho} \cdot N_{tg}
%\end{equation}

\item \textbf{$N_{e}$:} From $N_{e}^{r} = \sum_{i^{*}} \Delta C_{i^{*}}^{avg}(I_{i^{*}}^{avg}>I_{beam\_trip\_cut})$, since the charge is obtained from the average of four BCM monitor outputs ($u_{1},u_{3},d_{1}$ and $d_{3}$),the error is also averaged:
\begin{eqnarray*}
  \delta N_{e}^{r} &=& \sqrt{\frac{(\delta N_{e}^{r,d_{1}})^{2}+(\delta N_{e}^{r,d_{3}})^{2}+(\delta N_{e}^{r,u_{1}})^{2}+(\delta N_{e}^{r,u_{3}})^{2}}{4}}\\
                  &=& \sqrt{\frac{N_{e}^{r,d_{1}}+N_{e}^{r,d_{3}}+N_{e}^{r,u_{1}}+N_{e}^{r,u_{3}}}{4}}\\
                  &=& \frac{\sqrt{N_{e}^{r}}}{2} \\
\end{eqnarray*}
Hence,
\begin{equation}
  \delta N_{e} = \sqrt{\sum_{r}(\delta N_{e}^{r})^{2}}=\frac{1}{2}\sqrt{\sum_{r}N_{e}^{r}}=\frac{1}{2}\sqrt{N_{e}},
\end{equation}
where, $r$ means the run number.

\item \textbf{$Leve Time:$} Form $LT_{T_{i}} = \frac{N_{T_{i}}^{DAQ} PS_{T_{i}}}{N_{T_{i}}^{Scaler}}$,
\begin{equation}
  \delta LT^{r}_{T_{i}} = LT^{r}_{T_{i}} \cdot \sqrt{\frac{1}{N_{T_{i}}^{DAQ} PS_{T_{i}}}+\frac{1}{N_{T_{i}}^{Scaler}}},
\end{equation}
where there is one thing that confuses me, which is that whether I should multiply $PS_{T_{i}}$ in the first term or not. It won't give us problem so far since most of runs have PS equal to one.

\item \textbf{$N_{EX}:$} From  $N_{EX}^{r} = \frac{N_{T_{i}}^{r} \cdot PS_{T_{i}}^{r}}{LT_{T_{i}}^{r}}$ in one $x_{bj}$ bin, and $N_{EX}=\sum_{r}N_{EX}^{r}$, we have:
\begin{equation}
  \delta N_{EX}^{r} = N_{EX}^{r} \cdot \sqrt{\frac{1}{N_{T_{i}}^{r} PS_{T_{i}}} + (\frac{\delta LT_{T_{i}}^{r}}{LT_{T_{i}}^{r}})^{2} }, \delta N_{EX}=\sqrt{\sum_{r}(\delta N_{EX}^{r})^{2}}
\end{equation}

\item \textbf{$Y_{EX}:$} From Eq.\ref{eqyex},
\begin{equation}
  \delta Y_{EX} =  Y_{EX} \cdot \sqrt{(\frac{\delta N_{EX}}{N_{EX}})^{2}+(\frac{\delta N_{e}}{N_{e}})^{2}+(\frac{\delta\epsilon_{eff}}{\epsilon_{eff}})^{2}},
\end{equation}
where $\epsilon_{eff}$ is set to one and its statistic error and sysmatic error are set to zero and 1\%, respectively.

\item \textbf{$Y_{MC}:$} From Eq.\ref{eqymc},
\begin{equation}
  \delta Y_{MC} =  Y_{MC} \cdot \sqrt{(\frac{\delta N_{tg}}{N_{tg}})^{2}+(\frac{\delta\sum_{j\in i}}{\sum_{j\in i}})^{2}+(\frac{\delta N_{MC}^{gen}}{N_{MC}^{gen}})^{2}},
\end{equation}
where $\delta\sum_{j\in i} = \sum_{j\in i}\cdot\frac{1}{\sqrt{N_{MC}^{i}}}$, since it is sumarizing the cross section values of MC events ($N_{MC}^{i}$) in one bin. It might be wrong.

\item \textbf{$\sigma_{EX}^{born}:$} From Eq.\ref{eqxs},
  \begin{equation}
  \delta \sigma_{EX}^{born} = \sigma_{EX}^{born} \cdot \sqrt{(\frac{Y_{EX}}{Y_{EX}})^{2}+(\frac{Y_{MC}}{Y_{MC}})^{2}}
\end{equation}

\end{enumerate}
\end{document}

